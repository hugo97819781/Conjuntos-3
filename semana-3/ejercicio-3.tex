%%% FORMATO E IDIOMA %%%
    \documentclass[letterpaper,DIV=15,12pt]{scrartcl}
    \usepackage[spanish,mexico,shorthands=off,es-lcroman]{babel}
    \usepackage{scrlayer-scrpage}
        \clearpairofpagestyles
        \ihead{\footnotesize \textit{Teoría de Conjuntos III}}
        \ohead{\footnotesize \textit{2026-2}}
        \cfoot{\normalfont\thepage}
        \addtokomafont{title}{\bfseries \rmfamily}
        \setlength{\headsep}{5pt}
        \newpairofpagestyles{beginstyle}{
            \clearpairofpagestyles
            \KOMAoptions{headsepline=false}
            \cfoot{\footnotesize \pagemark}
            \cfoot{\normalfont\thepage}
        }
        \addtokomafont{section}{\raggedleft\rmfamily}
        \addtokomafont{subsection}{\raggedleft\rmfamily}
        \addtokomafont{subsubsection}{\raggedleft\rmfamily}
        \setlength{\headsep}{8pt}
        \setlength{\footskip}{20pt}
        \setlength{\textheight}{600pt}
%%% PAQUETES %%%
    \usepackage{ifthen}
    \usepackage[shortlabels]{enumitem}
        \setenumerate[1]{label=\MakeLowercase{\roman*}), ref=\roman*}
        \setenumerate[2]{label=\MakeLowercase{\alph*}), ref=\alph*}
    \usepackage{stix2}
%%%%%%%%%%%%%%%%%%%%%%%%%%%
%%% E J E R C I C I O S %%%
%%%%%%%%%%%%%%%%%%%%%%%%%%%
    \newcounter{Ejer}
    \newcounter{Pts}
    \setcounter{Ejer}{1}
    \setcounter{Pts}{1}
    \newcommand{\pts}{}
    \newenvironment{ejercicio}[1]{\noindent
        \ifthenelse{\equal{#1}{1}}{\renewcommand{\pts}{\textbf{(#1 pt)}}}{\renewcommand{\pts}{\textbf{(#1 pts)}}}\textbf{Ej. \theEjer} \pts\stepcounter{Ejer}}{}
%%% C O M A N D O S %%%
    \renewcommand{\emptyset}{\varnothing}
    \newcommand{\tq}{\mid}
%%%%%%%%%%%%%%%%%%%%%%%
%% D O C U M E N T O %%
%%%%%%%%%%%%%%%%%%%%%%%
\begin{document}

    %\thispagestyle{beginstyle}
       \begin{center}
            {\fontsize{30}{60}\rmfamily \textbf{Ejercicio semanal 3}} \\ \vspace{.2cm} Teoría de Conjuntos III, 2026-2
       \end{center}
        \begin{flushright}
            \footnotesize \hfill Profesor: Luis Jesús Turcio Cuevas.\\
            \hfill Ayudante: Hugo Víctor García Martínez.
        \end{flushright}
   
       \textit{\textbf{\textsc{Instrucciones.}} Esta tarea es \textbf{individual} y deberá ser entregada \textbf{presencial y personalmente} el día \textbf{lunes 2 de marzo} al inicio la clase.}\vspace{.3cm}

    \begin{ejercicio}{10}
        Sean $\mathscr{I}:=\{ [a,b) \mid a,b \in \mathbb{R} \}$ y $B:=\{ \bigcup F \mid F \subseteq \mathscr{I} \}$.
        \begin{enumerate}
          \item ¿Es $(B,\subseteq)$ un álgebra de Boole?
          \item Prueba que para ningún conjunto $X$ existe una biyección $f:\mathscr{P}(X) \to B$ de modo que para cualesquiera $A,B \subseteq X$ ocurra $A \subseteq B$ si y sólo si $f(A) \subseteq f(B)$. Es decir, muestra que $(\mathscr{P}(X),\subseteq)$ y $(B,\subseteq)$ no son órdenes isomorfos.
        \end{enumerate}
    \end{ejercicio}

\end{document}

