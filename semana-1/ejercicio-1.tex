%%% FORMATO E IDIOMA %%%
    \documentclass[letterpaper,DIV=15,12pt]{scrartcl}
    \usepackage[spanish,mexico,shorthands=off,es-lcroman]{babel}
    \usepackage{scrlayer-scrpage}
        \clearpairofpagestyles
        \ihead{\footnotesize \textit{Teoría de Conjuntos III}}
        \ohead{\footnotesize \textit{2026-2}}
        \cfoot{\normalfont\thepage}
        \addtokomafont{title}{\bfseries \rmfamily}
        \setlength{\headsep}{5pt}
        \newpairofpagestyles{beginstyle}{
            \clearpairofpagestyles
            \KOMAoptions{headsepline=false}
            \cfoot{\footnotesize \pagemark}
            \cfoot{\normalfont\thepage}
        }
        \addtokomafont{section}{\raggedleft\rmfamily}
        \addtokomafont{subsection}{\raggedleft\rmfamily}
        \addtokomafont{subsubsection}{\raggedleft\rmfamily}
        \setlength{\headsep}{8pt}
        \setlength{\footskip}{20pt}
        \setlength{\textheight}{600pt}
%%% PAQUETES %%%
    \usepackage{ifthen}
    \usepackage[shortlabels]{enumitem}
        \setenumerate[1]{label=\MakeLowercase{\roman*}), ref=\roman*}
        \setenumerate[2]{label=\MakeLowercase{\alph*}), ref=\alph*}
    \usepackage{stix2}
    \usepackage{amsmath}
    \usepackage{amsthm}
	\renewcommand{\qedsymbol}{$\blacksquare$}
	\addto\captionsspanish{\renewcommand{\proofname}{\normalfont\bfseries Demostración}}	
	\newenvironment{solucion}{\renewcommand{\qedsymbol}{$\boxtimes$}\begin{proof}[\normalfont\bfseries Solución]}{\end{proof}}
%%%%%%%%%%%%%%%%%%%%%%%%%%%
%%% E J E R C I C I O S %%%
%%%%%%%%%%%%%%%%%%%%%%%%%%%
    \newcounter{Ejer}
    \newcounter{Pts}
    \setcounter{Ejer}{1}
    \setcounter{Pts}{1}
    \newcommand{\pts}{}
    \newenvironment{ejercicio}[1]{\noindent
        \ifthenelse{\equal{#1}{1}}{\renewcommand{\pts}{\textbf{(#1 pt)}}}{\renewcommand{\pts}{\textbf{(#1 pts)}}}\textbf{Ej. \theEjer} \pts\stepcounter{Ejer}}{}
%%% C O M A N D O S %%%
    \renewcommand{\emptyset}{\varnothing}
    \newcommand{\tq}{\mid}
%%%%%%%%%%%%%%%%%%%%%%%
%% D O C U M E N T O %%
%%%%%%%%%%%%%%%%%%%%%%%
\begin{document}

    \thispagestyle{plain}
       \begin{center}
            {\fontsize{30}{60}\rmfamily \textbf{Ejercicio semanal 1}} \\ \vspace{.2cm} Teoría de Conjuntos III, 2026-2
       \end{center}
        \begin{flushright}
            \footnotesize \hfill Profesor: Luis Jesús Turcio Cuevas.\\
            \hfill Ayudante: Hugo Víctor García Martínez.
        \end{flushright}
   
       \textit{\textbf{\textsc{Instrucciones.}} Esta tarea es \textbf{individual} y deberá ser entregada \textbf{presencial y personalmente} el día \textbf{martes 11 de febrero} al inicio la clase.}\vspace{.3cm}

    \begin{ejercicio}{10}
        Sean $\mathbb{P}=(P,\leq)$ un forcing y $p,q \in P$. Considere:
        \[ D:=\{ x \in P \tq x \perp p \lor x \perp q \lor (x \leq p \land x \leq q) \} \]
    \end{ejercicio}
    \begin{enumerate}
        \item Pruebe que $D$ es denso en $\mathbb{P}$. \hfill \textbf{(5 pts)}
\item De un ejemplo de orden parcial $\mathbb{P}$ donde el conjunto $D$ sea más que numerable. \hfill \textbf{(5 pts)}
    \end{enumerate}
    \begin{proof}{\bfseries (\textit{i})}
        Sea $y \in P$ cualquiera. Sin pérdida de generalidad $y \notin D$, seguido de esto, se obtiene que $y \parallel p$ y existe $r \in P$ con $r \leq y,p$. Si $r \in D$, entonces $D \cap L(y) \neq \emptyset$. Por otro lado, $r \notin D$ implica que $r \parallel q$ y, así, existe $s \in P$ con $s \leq r,q$. Nótese que, como $r \leq y,p$, entonces $s \leq y$ y $s \leq p,q$, por tanto $s \in D$; en este caso, $D \cap L(y)$ tampoco es vacío. Por tanto, $D$ es denso.
    \end{proof}
    \begin{solucion}{\bfseries (\textit{ii})}
        Sea $\kappa \geq \aleph_1$, es vacuo que $\emptyset$ es un orden parcial en $\kappa$. Cualesquiera dos elementos (distintos) de $\mathbb{P}:=(\kappa,\emptyset)$ son incompatibles. Consecuentemente, para $p=q=0 \in \kappa$ se tiene que $D=\kappa$ es más que numerable.
    \end{solucion}
    
\end{document}

